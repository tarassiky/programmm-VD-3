\documentclass[a0paper,portrait]{tikzposter}

% ---------- Поддержка кириллицы ----------
\usepackage{fontspec}
\usepackage{polyglossia}
\setmainlanguage{russian}
\setotherlanguage{english}
\setmainfont{FreeSerif}
\setsansfont{FreeSans}
\setmonofont{FreeMono}

% ---------- Пакеты ----------
\usepackage{amsmath,amssymb,graphicx,tikz}

% ---------- Цветовая палитра (яркая, но не кричащая) ----------
\definecolor{mycyan1}{RGB}{180,240,240}   % светлая бирюза
\definecolor{mycyan2}{RGB}{110,200,210}   % насыщенная бирюза
\definecolor{myviolet}{RGB}{150,120,220}  % фиолетовый
\definecolor{mybg}{RGB}{235,250,250}      % нежный бирюзовый фон

% ---------- Оформление темы ----------
\usetheme{Simple}
\useblockstyle[
  roundedcorners=25,
  linewidth=2pt,
  titleinnersep=8pt,
  bodyinnersep=10pt
]{Default}

\definecolorstyle{brightstyle}{
  \definecolor{backgroundcolor}{named}{mybg}
  \definecolor{blocktitlebgcolor}{named}{mycyan2}
  \definecolor{blocktitlefgcolor}{named}{white}
  \definecolor{blockbodybgcolor}{named}{white}
  \definecolor{blockbodyfgcolor}{named}{black}
  \definecolor{titlefgcolor}{named}{black}
  \definecolor{framecolor}{named}{mybg}
  \definecolor{titlebgcolor}{named}{mybg}
}{}
\usecolorstyle{brightstyle}

% ---------- Заголовок ----------
\title{Визуализация данных в Python с использованием библиотеки Matplotlib}
\author{Тарасова Дарья Юрьевна}
\institute{РГПУ им. А. И. Герцена}

\titlegraphic{
\vspace{-2cm}
\centering
\includegraphics[width=10cm]{logo.png}\\[1cm]
}

\begin{document}
\maketitle

% ---------- Аннотация ----------
\block{Аннотация}{
Работа посвящена изучению инструментов визуализации данных в языке программирования Python
с использованием библиотеки \textbf{Matplotlib}.
Рассмотрены основные типы графиков, настройки отображения и приёмы повышения информативности диаграмм.
Созданы примеры визуализаций с применением цветовых схем и пользовательских параметров.
}

% ---------- Введение ----------
\block{Введение}{
Визуализация данных является важной частью анализа информации.
Python предоставляет широкий набор инструментов для построения графиков,
среди которых одной из наиболее популярных библиотек является \textbf{Matplotlib}.
Она позволяет создавать линейные графики, гистограммы, круговые диаграммы,
а также настраивать цвета, подписи и стиль отображения.
В данной работе рассматриваются базовые возможности библиотеки.
}

% ---------- Методы ----------
\block{Методы}{
Для построения визуализаций использовалась библиотека \textbf{Matplotlib}:

\begin{itemize}
\item построение линейных графиков: \verb|plt.plot()|;
\item построение гистограмм: \verb|plt.hist()|;
\item круговые диаграммы: \verb|plt.pie()|;
\item настройка цвета, толщины линий, подписей и сетки;
\item сохранение изображений в форматах PNG и SVG.
\end{itemize}

\vspace{8pt}
\begin{center}
\begin{tikzpicture}[scale=0.9]
  % Основа графика
  \draw[thick, ->] (0,0) -- (6,0);
  \draw[thick, ->] (0,0) -- (0,4);

  % Линия графика
  \draw[very thick, myviolet] (0,1) -- (1,2.2) -- (2,1.7) -- (3,3.2) -- (4,2.8) -- (5,3.6);

  \node at (3,-0.7) {\small Пример линейного графика};
\end{tikzpicture}
\end{center}
}

% ---------- Результаты ----------
\block{Результаты}{
В процессе работы были созданы примеры основных типов визуализаций:

\begin{itemize}
\item линейный график изменения параметра во времени;
\item гистограмма распределения случайных данных;
\item круговая диаграмма для демонстрации долей категорий;
\item сравнение стилей оформления и цветовых схем.
\end{itemize}

Были изучены особенности применения сетки, легенд и адаптивного масштаба,
а также параметры сохранения изображений (\verb|dpi|, формат, прозрачность).
}

% ---------- Выводы ----------
\block{Выводы}{
\begin{itemize}
\item изучены базовые возможности Matplotlib для визуализации данных;
\item рассмотрены типы графиков и особенности их применения;
\item созданы примеры графиков с пользовательскими настройками;
\item показано, что Matplotlib позволяет формировать информативные визуализации
для анализа данных и презентаций.
\end{itemize}
}

% ---------- Список литературы ----------
\block{Список литературы}{
1. Hunter J. D. \emph{Matplotlib: A 2D Graphics Environment}, 2007.\\
2. VanderPlas J. \emph{Python Data Science Handbook}. O'Reilly, 2016.\\
3. Официальная документация Matplotlib — \texttt{https://matplotlib.org}.\\
4. Макаров А. \emph{Python для анализа данных}. — М.: Питер, 2023.
}

\end{document}