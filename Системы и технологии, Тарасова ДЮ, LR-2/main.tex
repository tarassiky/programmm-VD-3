\documentclass[11pt,a4paper,sans]{moderncv}

% ===== НАСТРОЙКИ ШРИФТОВ И ЯЗЫКА =====
\usepackage{fontspec}
\defaultfontfeatures{Renderer=Basic, Ligatures=TeX}
\newfontfamily\cyrillicfonttt{CMU Typewriter Text}
\newfontfamily\cyrillicfont{CMU Sans Serif}
\setmainfont{CMU Sans Serif}
\setsansfont{CMU Sans Serif}
\setmonofont{CMU Typewriter Text}

\usepackage{polyglossia}
\setdefaultlanguage{russian}
\setotherlanguages{english}

% ===== СТИЛЬ MODERNCV =====
\moderncvstyle{fancy} % casual, classic, banking, oldstyle, fancy
\moderncvcolor{purple} % blue, orange, green, red, purple, grey, black

% ===== НАСТРОЙКИ ПОЛЕЙ =====
\usepackage[scale=0.75]{geometry}

% ===== ПЕРСОНАЛЬНАЯ ИНФОРМАЦИЯ =====
\name{Даша}{Тарасова}
\title{Разработчик программного обеспечения}
\address{г. Москва, Россия}{}{}
\phone[mobile]{+7~(999)~123~45~67}
\email{dasha.tarasova@email.com}
\homepage{www.tarasova.com}
\social[linkedin]{dasha-tarasova}
\social[github]{tarasova}
\photo[64pt][0.4pt]{imagge.jpg} 
\quote{Ответственный и целеустремленный разработчик с опытом работы...}

\begin{document}

\makecvtitle

% ===== ОПЫТ РАБОТЫ =====
\section{Опыт работы}
\cventry{2022--наст. время}{Разработчик ПО}{ООО "ТехноСофт"}{Москва}{}{
\begin{itemize}
\item Разработка веб-приложений на Java и Spring Framework
\item Оптимизация производительности приложений
\item Участие в код-ревью и менторинг junior-разработчиков
\end{itemize}}

\cventry{2020--2022}{Junior разработчик}{ООО "ИТ-Решения"}{Москва}{}{
\begin{itemize}
\item Разработка компонентов пользовательского интерфейса
\item Тестирование и отладка приложений
\item Участие в планировании спринтов
\end{itemize}}

% ===== ОБРАЗОВАНИЕ =====
\section{Образование}
\cventry{2016--2020}{Бакалавр}{Московский технический университет}{Москва}{}{
\begin{itemize}
\item Направление: Информатика и вычислительная техника
\item Средний балл: 4.8
\item Тема диплома: "Разработка системы управления проектами"
\end{itemize}}

\cventry{2020--2022}{Магистр}{Московский технический университет}{Москва}{}{
\begin{itemize}
\item Направление: Программная инженерия
\item Средний балл: 5.0
\end{itemize}}

% ===== НАВЫКИ =====
\section{Навыки}
\subsection{Программирование}
\cvitem{Продвинутый}{Java, Spring Framework, SQL, Git}
\cvitem{Средний}{Python, JavaScript, HTML/CSS}
\cvitem{Базовый}{Docker, Kubernetes, React}

\subsection{Языки}
\cvitem{Русский}{Родной}
\cvitem{Английский}{B2 (Upper Intermediate)}
\cvitem{Немецкий}{A2 (Basic)}

% ===== ПРОЕКТЫ =====
\section{Проекты}
\cventry{2023}{Система управления задачами}{Личный проект}{}{}{
\begin{itemize}
\item Разработка full-stack приложения для управления проектами
\item Использованы: Java, Spring Boot, React, PostgreSQL
\item Реализована JWT аутентификация и REST API
\end{itemize}}

\cventry{2022}{Мобильное приложение для фитнеса}{Учебный проект}{}{}{
\begin{itemize}
\item Разработка кроссплатформенного приложения на Flutter
\item Интеграция с Google Fit и Apple HealthKit
\end{itemize}}

% ===== СЕРТИФИКАТЫ =====
\section{Сертификаты}
\cvitem{2023}{Oracle Certified Professional, Java SE 11 Developer}
\cvitem{2022}{AWS Certified Cloud Practitioner}
\cvitem{2021}{Scrum Master Certified (SMC)}

% ===== ДОПОЛНИТЕЛЬНАЯ ИНФОРМАЦИЯ =====
\section{Дополнительно}
\cvitem{Хобби}{Программирование, чтение технической литературы, путешествия}
\cvitem{Волонтерство}{Помощь в организации IT-конференций, менторинг студентов}

\end{document}