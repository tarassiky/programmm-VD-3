\documentclass[14pt]{extreport}

% ===== ЗАДАНИЕ 2: Настройка документа =====
\usepackage{fontspec}
\usepackage{polyglossia}
\setmainlanguage{russian}
\setotherlanguage{english}

% Установка шрифтов с поддержкой кириллицы
\setmainfont{DejaVu Serif} % Шрифт с поддержкой кириллицы
\newfontfamily\cyrillicfont{DejaVu Serif} % Явно указываем шрифт для кириллицы

% ===== ЗАДАНИЕ 7: Настройка полей по ГОСТ =====
\usepackage[a4paper,left=30mm,right=10mm,top=20mm,bottom=20mm]{geometry}
\usepackage{indentfirst}
\setlength{\parindent}{1.25cm}
\setlength{\parskip}{0pt}
\linespread{1.5}

% ===== Дополнительные пакеты для заданий =====
\usepackage{graphicx} % Для изображений
\usepackage{amsmath}  % Для математических формул
\usepackage{hyperref} % Для гиперссылок
\usepackage{xcolor}   % Для работы с цветами

% ===== Настройка нумерации страниц =====
\usepackage{fancyhdr}
\pagestyle{fancy}
\fancyhf{}
\fancyfoot[C]{\thepage}
\renewcommand{\headrulewidth}{0pt}

\begin{document}

% ===== ЗАДАНИЕ 3: Титульная страница =====
\title{Реферат на тему: \\[0.5cm] \textbf{Искусственный интеллект в современном мире}}
\author{Студент: Тарасова Дарья Юрьевна \\ Группа: 1.2}
\date{\today}
\maketitle

% ===== Оглавление =====
\tableofcontents

% ===== ЗАДАНИЕ 4: Разделы и подпункты =====
\chapter{Введение}
В данном реферате рассматриваются основные аспекты искусственного интеллекта и его влияние на современное общество. Искусственный интеллект становится неотъемлемой частью нашей жизни, проникая в различные сферы деятельности.

\chapter{Основная часть - 1}
\section{Раздел 1.1: История развития ИИ}
Развитие искусственного интеллекта началось в середине XX века. \textbf{Первые исследования} в этой области проводились в 1950-х годах.

\subsection{Подраздел 1.1.1: Основные вехи}
\textit{Важные этапы} развития искусственного интеллекта включают создание первых нейронных сетей и экспертных систем.

\subsection{Подраздел 1.1.2: Современные тенденции}
\underline{Современный ИИ} характеризуется использованием глубокого обучения и больших данных.

\subsubsection{Подподраздел 1.1.2.1: Глубокое обучение}
Глубокое обучение революционизировало подход к решению сложных задач.

\section{Раздел 2: Применение ИИ}
Искусственный интеллект находит применение в различных областях:

\begin{itemize}
\item Медицина и диагностика заболеваний
\item Автомобилестроение (беспилотные автомобили)
\item Финансовый сектор
\item Образование
\end{itemize}

% ===== ЗАДАНИЕ 8: Вставка изображения =====
\begin{figure}[h]
\centering
\includegraphics[width=0.6\textwidth]{example-image} % Встроенная тестовая картинка
\caption{Схема работы нейронной сети}
\label{fig:neural_network}
\end{figure}

Как показано на рисунке \ref{fig:neural_network}, современные нейронные сети состоят из входного, скрытых и выходного слоев.

% ===== ЗАДАНИЕ 9: Создание таблицы =====
\begin{table}[h]
\centering
\begin{tabular}{|p{2cm}|p{4cm}|p{5cm}|} % Фиксированные ширины колонок
\hline
\textbf{Год} & \textbf{Событие} & \textbf{Значение} \\
\hline
1956 & Конференция в Дартмуте & Рождение ИИ как науки \\
\hline
1997 & Deep Blue побеждает Каспарова & Прорыв в игровом ИИ \\
\hline
2016 & AlphaGo побеждает чемпиона & Достижения в глубоком обучении \\
\hline
\end{tabular}
\caption{Ключевые события в истории ИИ}
\label{tab:ai_history}
\end{table}

% ===== ЗАДАНИЕ 10: Вставка формул =====
\chapter{Математические основы}
Основные математические концепции ИИ включают:

\begin{equation}
y = \sigma(wx + b)
\end{equation}

где:
\begin{align}
y & \text{ - выход нейрона} \nonumber \\
x & \text{ - входные данные} \nonumber \\
w & \text{ - весовой коэффициент} \nonumber \\
b & \text{ - смещение} \nonumber
\end{align}

% ===== ЗАДАНИЕ 6: Базовые элементы форматирования =====
\section{Форматирование текста}
В тексте реферата используются различные стили форматирования:

\textbf{Жирный текст} используется для выделения ключевых терминов и определений. \textit{Курсивный текст} применяется для выделения названий и цитат. \underline{Подчёркнутый текст} может использоваться для особого акцента, хотя применяется реже.

% ===== ЗАДАНИЕ 11: Гиперссылки =====
\section{Дополнительные ресурсы}
Более подробную информацию можно найти на сайте \href{https://example.com}{Пример образовательного ресурса}.

\chapter{Заключение}
Искусственный интеллект продолжает развиваться и оказывает значительное влияние на различные аспекты современной жизни. Дальнейшее развитие этой технологии открывает новые возможности и ставит важные этические вопросы.

% ===== ЗАДАНИЕ 5: Список литературы =====
\begin{thebibliography}{9}
\item Рассел С., Норвиг П. Искусственный интеллект: современный подход. - М.: Вильямс, 2020.
\item Гудфеллоу Я., Бенджио И., Курвилль А. Глубокое обучение. - М.: ДМК Пресс, 2018.
\item Smith J. AI in Modern Society. Journal of Artificial Intelligence, 2022.
\end{thebibliography}

\end{document}